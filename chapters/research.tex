\section{Research}

\subsection{Dataset}

The dataset is available on the UCI website \url{https://archive.ics.uci.edu/ml/datasets/wine+quality} There are 2 datasets. One for white wine and one for red wine. This paper focuses on the white wine dataset. The dataset holds the following features

\begin{enumerate}
  \item Fixed acidity 
  \item Volatile acidity
  \item Citric acid 
  \item Residual sugar 
  \item Chlorides
  \item Free sulfur dioxide 
  \item Total sulfur dioxide 
  \item Density
  \item PH
  \item Sulphates
  \item Alcohol
  \item Quality (score between 0 and 10)
\end{enumerate}

We will be trying to predict the quality feature. This ranges from 0 to 10, 0 being the worst and 10 being the best quality wine.

\subsection{Outlier Detection}

Outlier detection is an important part of pre-processing data. Outliers have the ability to skew resulting predictive models by creating noise or abnormalities, therefore it is important that we can effectively find and resolve these outliers.

This paper uses a simple visual method for outlier detection called a boxplot.
We use a quartile-based boxplot, which uses the lower quartile (Ql, 25th percentile), the median (x̃  or Q2, 50th percentile), the upper quartile (Q3, 75th percentile), and the interquartile range (IQR=Q3-Q1). Then Boxplot outlier rule can be expressed as\cite{6520712}

\begin{equation}
x_{i} > Q_{3}+1.5IQR\ \bigcup \ x_{i} < Q_{1}-1.5IQR
\end{equation}

\subsection{Feature Selection}

This paper effectively uses heatmaps to reduce the number of features in the dataset. During experimentation it was found that residual sugar and density had a high degree of correlation. This presented the opportunity to remove one of the features. In this case it was decided to drop the residual sugar feature from the dataset.

\subsection{Logistic Regression}

