\section{Research}

\subsection{Outlier Detection}

Outlier detection is an important part of pre-processing data. Outliers have the ability to skew resulting predictive models by creating noise or abnormalities, therefore it is important that we can effectively find and resolve these outliers.

This paper uses a simple visual method for outlier detection called a boxplot.
We use a quartile-based boxplot, which uses the lower quartile (Ql, 25th percentile), the median (x̃  or Q2, 50th percentile), the upper quartile (Q3, 75th percentile), and the interquartile range (IQR=Q3-Q1). Then Boxplot outlier rule can be expressed as\cite{6520712}

\begin{equation}
x_{i} > Q_{3}+1.5IQR\ \bigcup \ x_{i} < Q_{1}-1.5IQR
\end{equation}

\subsection{Feature Selection}

This paper effectively uses heatmaps to reduce the number of features in the dataset. During experimentation it was found that residual sugar and density had a high degree of correlation. This presented the opportunity to remove one of the features. In this case it was decided to drop the residual sugar feature from the dataset.

\subsection{Classification Models}

This project uses Logistic Regression, Decision Tree and Random Forest classifiers to try and create a classification model for the quality of wine

