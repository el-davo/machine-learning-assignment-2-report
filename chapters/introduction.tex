\section{Introduction}

Wine quality tends to be very subjective depending on the taster. To circumvent this, wine is generally judged based upon its phsyiological and chemical make up. If we can create a classification model based on a number of chemical compounds then it would help wine makers create better wines. 

The dataset used in this project is available on the UCI website \url{https://archive.ics.uci.edu/ml/datasets/wine+quality} There are 2 datasets. One for white wine and one for red wine. This paper focuses on the white wine dataset. The dataset holds the following features

\begin{enumerate}
  \item Fixed acidity 
  \item Volatile acidity
  \item Citric acid 
  \item Residual sugar 
  \item Chlorides
  \item Free sulfur dioxide 
  \item Total sulfur dioxide 
  \item Density
  \item PH
  \item Sulphates
  \item Alcohol
  \item Quality (score between 0 and 10)
\end{enumerate}

This project aims to create a classification model for predicting the quality of white wine. This ranges from 0 to 10, 0 being the worst and 10 being the best quality wine.

There have been a number of studies already carried out on this dataset, in particular, A classification approach with different feature sets to predict the quality of different types of wine using machine learning techniques\cite{8323674} and Assessing wine quality using a decision tree\cite{7302752}

